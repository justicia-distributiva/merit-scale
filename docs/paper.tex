% Options for packages loaded elsewhere
\PassOptionsToPackage{unicode}{hyperref}
\PassOptionsToPackage{hyphens}{url}
\PassOptionsToPackage{dvipsnames,svgnames*,x11names*}{xcolor}
%
\documentclass[
]{article}
\usepackage{lmodern}
\usepackage{setspace}
\usepackage{amssymb,amsmath}
\usepackage{ifxetex,ifluatex}
\ifnum 0\ifxetex 1\fi\ifluatex 1\fi=0 % if pdftex
  \usepackage[T1]{fontenc}
  \usepackage[utf8]{inputenc}
  \usepackage{textcomp} % provide euro and other symbols
\else % if luatex or xetex
  \usepackage{unicode-math}
  \defaultfontfeatures{Scale=MatchLowercase}
  \defaultfontfeatures[\rmfamily]{Ligatures=TeX,Scale=1}
\fi
% Use upquote if available, for straight quotes in verbatim environments
\IfFileExists{upquote.sty}{\usepackage{upquote}}{}
\IfFileExists{microtype.sty}{% use microtype if available
  \usepackage[]{microtype}
  \UseMicrotypeSet[protrusion]{basicmath} % disable protrusion for tt fonts
}{}
\makeatletter
\@ifundefined{KOMAClassName}{% if non-KOMA class
  \IfFileExists{parskip.sty}{%
    \usepackage{parskip}
  }{% else
    \setlength{\parindent}{0pt}
    \setlength{\parskip}{6pt plus 2pt minus 1pt}}
}{% if KOMA class
  \KOMAoptions{parskip=half}}
\makeatother
\usepackage{xcolor}
\IfFileExists{xurl.sty}{\usepackage{xurl}}{} % add URL line breaks if available
\IfFileExists{bookmark.sty}{\usepackage{bookmark}}{\usepackage{hyperref}}
\hypersetup{
  pdftitle={Measuring Perceptions and Preferences for Meritocracy},
  pdfauthor={Juan Carlos Castillo, Julio Iturra \& Francisco Meneses},
  colorlinks=true,
  linkcolor=blue,
  filecolor=Maroon,
  citecolor=Blue,
  urlcolor=Blue,
  pdfcreator={LaTeX via pandoc}}
\urlstyle{same} % disable monospaced font for URLs
\usepackage[margin=0.78in]{geometry}
\usepackage{longtable,booktabs}
% Correct order of tables after \paragraph or \subparagraph
\usepackage{etoolbox}
\makeatletter
\patchcmd\longtable{\par}{\if@noskipsec\mbox{}\fi\par}{}{}
\makeatother
% Allow footnotes in longtable head/foot
\IfFileExists{footnotehyper.sty}{\usepackage{footnotehyper}}{\usepackage{footnote}}
\makesavenoteenv{longtable}
\setlength{\emergencystretch}{3em} % prevent overfull lines
\providecommand{\tightlist}{%
  \setlength{\itemsep}{0pt}\setlength{\parskip}{0pt}}
\setcounter{secnumdepth}{5}
\usepackage{times}
\usepackage{caption}
\captionsetup[figure, table]{labelfont={bf},labelformat={default},labelsep=period}
\usepackage{graphicx}
\usepackage{float}
\usepackage{booktabs}
\usepackage{longtable}
\usepackage{array}
\usepackage{multirow}
\usepackage{wrapfig}
\usepackage{float}
\usepackage{colortbl}
\usepackage{pdflscape}
\usepackage{tabu}
\usepackage{threeparttable}

\title{Measuring Perceptions and Preferences for Meritocracy}
\author{Juan Carlos Castillo, Julio Iturra \& Francisco Meneses}
\date{}

\begin{document}
\maketitle
\begin{abstract}
Economic and social inequality have raised growing concerns and crises across societies. One of social science concepts associated to the maintenance of inequality is the belief in meritocracy, which would legitimize economic differences based on effort and talent. Despite its wide use, empirical research on meritocracy is something relatively novel. A number of studies have relied mostly in secondary data to operationalize meritocracy, with a large variation in the use and interpretation of survey items. Starting from a review of studies that measure meritocracy, this article identifies a series of drawbacks and inconsistencies within and between studies regarding the conceptualization and measurement of meritocracy. Based on this critical analysis, we propose an items battery called \emph{Perceptions and preferences for meritocracy scale}, which is tested with confirmatory factor analysis with data from an online survey study (N=2,141). The results support the proposed conceptual structure which not only distinguishes between perceptions and preferences, but also between meritocratic and non-meritocratic dimensions. The discussion highlights the relevance of considering these different dimensions in order to advance in the study of meritocracy.
\end{abstract}

\setstretch{1.5}
\hypertarget{part-introducciuxf3n}{%
\part*{Introducción}\label{part-introducciuxf3n}}
\addcontentsline{toc}{part}{Introducción}

La desigualdad económica se ha vuelto un tema que genera creciente preocupación y malestar alrededor del mundo. Esto se ha expresado en una serie de protestas como la emblemática ``occupy wall street'' el año 2011, así como también en una serie de análisis críticos respecto del desarrollo del capitalismo y sus consecuencias ({\textbf{???}}). En este contexto, el estudio de las visiones, preferencias y percepciones respecto de la desigualdad han adquirido relevancia en las ciencias sociales, en temas como las preferencias redistributivas (Alesina and Angeletos \protect\hyperlink{ref-alesina_Fairness_2005}{2005}; Dimick, Rueda, and Stegmueller \protect\hyperlink{ref-dimick_Models_2018}{2018}) la legitimación de la desigualdad económica (Schröder \protect\hyperlink{ref-schroder_Income_2017}{2017}) y el funcionamiento de la meritocracia ({\textbf{???}}; {\textbf{???}}; {\textbf{???}}).

En general, la meritocracia se define como un sistema de distribución de recursos y recompensas basados en el mérito individual, que en su concepción original es una suma de talento y esfuerzo (Young \protect\hyperlink{ref-young_rise_1962}{1962}). Esta concepción tradicional de mérito pone en un lugar secundario la posible interferencia de factores estructurales o no meritocráticos como la herencia, los contactos personales, o la suerte (Breen and Goldthorpe \protect\hyperlink{ref-breenClassInequalityMeritocracy1999}{1999}; Saunders \protect\hyperlink{ref-saundersMightBritainBe1995}{1995}; Yair \protect\hyperlink{ref-yairMeritocracy2007}{2007}; Land \protect\hyperlink{ref-landWeSatTable2006}{2006}; Young \protect\hyperlink{ref-youngRiseMeritocracy1994}{1994}). Una serie de estudios han realizado críticas a la realización de este estándar moral de distribución, planteando que es una promesa incumplida dada la influencia preponderante de otros elementos más allá del mérito en el estatus individual ({\textbf{???}}; Arrow, Bowles, and Durlauf \protect\hyperlink{ref-arrow_meritocracy_2000}{2000}; Goldthorpe \protect\hyperlink{ref-goldthorpe_myth_2003}{2003}; Markovits \protect\hyperlink{ref-markovits_Meritocracy_2019}{2019}). Por otro lado, desde la psicología social y la sociología se han estudiado las características y consecuencias de las creencias en la meritocracia, en general basados en la hipótesis que mayor creencia en la meritocracia lleva a una mayor legitimación de las desigualdades ({\textbf{???}}; {\textbf{???}}; {\textbf{???}}; Hadjar \protect\hyperlink{ref-hadjar_meritokratie_2008}{2008}; Madeira et al. \protect\hyperlink{ref-MadeiraPrimesConsequencesSystematic2019}{2019}).

Debido al rol que cumplen las creencias meritocráticas dentro del pensamiento neoliberal ({\textbf{???}}), han surgido múltiples investigaciones que evalúan la relación entre creencias meritocráticas y diversos ámbitos sociales de la actualidad. Por ejemplo, se han desarrollado estudios que vinculan la meritocracia al reforzamiento de estereotipos socioeconómicos, de género y de etnias (Madeira et al. \protect\hyperlink{ref-MadeiraPrimesConsequencesSystematic2019}{2019}; {\textbf{???}}; {\textbf{???}}), así como también líneas de investigación que evalúan el efecto de las creencias meritocráticas en el contexto educativo ({\textbf{???}}; {\textbf{???}}; {\textbf{???}}) y en el contexto organizacional de las empresas ({\textbf{???}}; {\textbf{???}}).

Para poder dar cuenta de los niveles de creencia en la meritocracia los estudios a la fecha generalmente han utilizado algunos indicadores de encuestas ya existentes, y en el menor de los casos se han creado instrumentos ad-hoc. Sin embargo, y como mostraremos más adelante, las formas de medición de meritocracia varían extremadamente entre estudios. Muchas veces fenómenos similares se asocian a indicadores distintos, y también ocurre que fenómenos distintos son medidos con indicadores similares, todo lo cual hace dificulta la comparabilidad entre estudios y el poder avanzar en la comprensión y estudio de la meritocracia.

Basados en el análisis crítico de las formas de medición de meritocracia a la fecha, el presente artículo propone un instrumento para medir y relacionar dos aspectos claves en el estudio de la meritocracia: percepciones y preferencias. Además, como un segundo eje de análisis considera la generación de indicadores respecto de aspectos meritocráticos y anti-meritocráticos, demostrando que no son los dos polos de un mismo continuo como muchos estudios anteriores parecen sugerir. La propuesta de medición además está orientada a generar un instrumento lo más breve posible de manera que pueda ser utilizado en encuestas de opinión pública y así ser asociado a otros fenómenos sociales.

\hypertarget{la-mediciuxf3n-de-los-aspectos-subjetivos-de-la-meritocracia}{%
\section{La medición de los aspectos subjetivos de la meritocracia}\label{la-mediciuxf3n-de-los-aspectos-subjetivos-de-la-meritocracia}}

A continuación se presenta una revisión de una serie de investigaciones que se han abocado al estudio de la meritocracia y que para ello han hecho una propuesta de medición. El primer eje de análisis tiene que ver con el uso del concepto ``creencias'' para referir a distintos aspectos subjetivos relacionados con meritocracia. El segundo eje tiene que ver con el uso de indicadores sobre aspectos anti-meritocráticos como el polo opuesto de los meritocráticos.

\hypertarget{blackbox}{%
\subsection*{The black-box of meritocratic beliefs}\label{blackbox}}
\addcontentsline{toc}{subsection}{The black-box of meritocratic beliefs}

Several approaches to the empirical study of meritocracy based on public opinion surveys make reference to the concept of \emph{beliefs}, but behind this concept there are usually different meanings and operationalizations. To illustrate this point in the following we will start with the proposal from a recent paper by Mijs (\protect\hyperlink{ref-mijs_paradox_2019}{2019}), which we will take as a reference to discuss previous studies.

The meritocratic beliefs' definition of Mijs is the following: ``when I discuss meritocracy beliefs, I am referring to citizens' belief in the importance of hard work relative to structural factors.'' (Mijs \protect\hyperlink{ref-mijs_paradox_2019}{2019}, pg.9). In the operationalization, this is associated with the following indicator: ``how important you think it is for getting ahead in life: (a) hard work'', scored in a 1 to 5 likert scale. There are several assumptions behind this decision that are worth discussing:

\begin{enumerate}
\def\labelenumi{\alph{enumi}.}
\tightlist
\item
  \emph{Dimensionality}
\end{enumerate}

The item used by Mijs is part of an items' battery present in several international surveys, usually called ``reasons to get ahead''. This battery presents a series of indicators related to what people consider important to get ahead in life: hard work, education, ambition, wealthy family, right connections, religion, race and gender. Therefore, for Mijs other aspects as education, that could be associated to talent, are not meritocratic. As he points out: ``Hard work is arguably the most meritocratic part of Michael Young's equation, `Merit = Intelligence + Effort', for the simple fact that intelligence itself is conditioned by a nonmeritocratic factor: who your parents happen to be'' (p.5).

In Mijs' proposal we can observe a couple of strong assumptions: effort would not depend on parents influence, and talent is not meritocratic (contrary to Michael Youngs original conceptualization). The problem of whether talent is or not meritocratic is something that actually should (and can) be tested, it cannot be simply ruled out particularly in empirical research. This conceptual and measurement issue it is possible to find in other studies that assume that effort is the main and only aspect of meritocracy ({\textbf{???}}; {\textbf{???}}; {\textbf{???}}; {\textbf{???}}).

\emph{b. Beliefs}

The ``resons to get ahead'' battery refers to ``how important you think it is'', considered by Mijs as a belief in meritocracy for the item regarding effort. Nevertheless, another version of this same battery in the survey asks about ``how important you think it \emph{should} be''. Which one of both is a ``belief''? According to ({\textbf{???}}) : ``Perceptions refer to subjective estimates of existing inequality (i.e.~thoughts about what is). Beliefs are here defined as normative ideas about just inequality (i.e.~thoughts about what should be)''(p.359). Therefore, the referred paper would be actually using the term beliefs referring to perceptions. The same occurs in Reynolds and Xian (\protect\hyperlink{ref-reynolds_perceptions_2014}{2014}), who explicitly use the term beliefs to talk about perceptions, whereas other authors use different terms as attitudes (Kunovich and Slomczynski (\protect\hyperlink{ref-kunovich_systems_2007}{2007})). The first attempt to shed light on this confusion was made by Duru-Bellat and Tenret (\protect\hyperlink{ref-duru-bellat_whos_2012}{2012}), who used the item ``how important should the number of years spent in education and training be in deciding how much money people ought to earn?'' for ``desired'' meritocracy (beliefs), whereas for ``perceived'' meritocracy they use two items: ``Would you say that in your country, people are rewarded for their efforts?'' and ``\ldots{} people are rewarded for their skills?''.

Is the belief in meritocracy a perception or a desire/preference? In order to expand the analitical conceptual framework, we believe that both dimensions should be included in the analysis, as proposed by Duru-Bellat and Tenret (\protect\hyperlink{ref-duru-bellat_whos_2012}{2012}). This opens up possibilities of analyzing whether perceptions and preferences are actually the same (i.e.~correlation close to 1) or they are different aspects of the same phenomenon. As Son Hing et al. (\protect\hyperlink{ref-son_hing_merit_2011-1}{2011}) have pointed out, ``People can believe that outcomes ought to be distributed on the basis of merit and yet vary in their perceptions of whether this is how society currently operates'' (p.~435). In other words, normative beliefs should be interpreted in the context of perception: a large normative belief in meritocracy certainly means something totally different for someone perceiving high meritocracy than for someone perceiving low meritocracy. In order to avoid the confusion generated by the term ``belief'', we propose the terms meritocratic preferences and meritocratic perceptions, as they better reflect the two dimension under scrutiny.

\emph{c.~Non-meritocratic aspects}

Mijs (\protect\hyperlink{ref-mijs_paradox_2019}{2019}) makes reference to some non-meritocratic aspects as talent, which is ruled out of the operationalization of meritocracy. A different approach was followed by Kunovich and Slomczynski (\protect\hyperlink{ref-kunovich_systems_2007}{2007}), who decide to include some non-meritocratic elements. Using the items' battery listing a number of reasons about ``How important should be in deciding pay\ldots{}'' (as Duru-Bellat and Tenret (\protect\hyperlink{ref-duru-bellat_whos_2012}{2012}) for desired meritocracy), he decided that reasons as education and responsibility are meritocratic and pointed 1 if considered essential, whereas reasons such as having a family and children were pointed 1 if they were considered ``not important at all'' (i.e.~reverse coded). A similar approach was taken by Newman, Johnston, and Lown (\protect\hyperlink{ref-newman_false_2015}{2015}), reverse-coding non-meritocratic items, similarly to what occurs with the ``Preference for the Merit Principle Scale'' (Davey et al. \protect\hyperlink{ref-davey_preference_1999}{1999}).

The assumption that meritocratic and non-meritocratic elements are the poles of the same continuum was analyzed by Reynolds and Xian (\protect\hyperlink{ref-reynolds_perceptions_2014}{2014}) using the same ``get ahead'' perceptions' battery items. They consider education, ambition and hard work as meritocratic and other reasons such as wealthy family and right connections and non-mertitocratic. Nevertheless, despite making this distinction the author ends up subtracting one dimension from the other, assuming that they are two poles of the same continuum as Kunovich and Slomczynski (\protect\hyperlink{ref-kunovich_systems_2007}{2007}) did. Still, taking into account this research perspective, we suggest that non-meritocratic aspects should be part of a meritocratic measurement but taken independently and not adding or subtracting from meritocratic ones unless it is empirically proved that they belong to the same conceptual dimension.

\emph{d.~Accounting for measurement error}

Finally, most of the studies in meritocracy so far have not incorporated the issue of measurement error ({\textbf{???}}), using single indicators and/or simple average indexes for measuring meritocracy. Such strategy assumes that the construct is measured perfectly by the indicators chosen, going as far as proposing that "\ldots{} In choosing this strategy of index construction, we argue that \emph{support for meritocracy is not a latent variable} (Kunovich and Slomczynski \protect\hyperlink{ref-kunovich_systems_2007}{2007}, 653--54). Some advances were done by Reynolds and Xian (\protect\hyperlink{ref-reynolds_perceptions_2014}{2014}) by doing a principal component analysis of meritocratic and non-meritocratic dimensions, but somewhat contradictorily they end up in a sum index despite proving multidimensionality.

\hypertarget{instrumentprop}{%
\subsection*{An instrument proposal}\label{instrumentprop}}
\addcontentsline{toc}{subsection}{An instrument proposal}

Based on the previous limitations and confussions in the measurement of meritocracy presented in the previous section, in this paper we propose and test an instrument with the following characteristics:

\begin{itemize}
\item
  Multidimensional, incorporating previous distinctions between preferences and perceptions as well as between meritocratic and non-meritocratic aspects.
\item
  Multiple indicators for each dimension, in order to account for measurement error in a confirmatory factor analysis context.
\item
  Based on previous indicators as far as possible, for the sake of comparability between studies
\item
  Brief, as to be used in a regular survey. In this point it differs for instance from the proposal of the ``Preference for the Merit Principle Scale'' (Davey et al. \protect\hyperlink{ref-davey_preference_1999}{1999}), as they use 15 items just for one dimension (besides the problem of reverse-coding non-meritocratic items).
\end{itemize}

The proposed measurement framework is depicted in Figure \ref{fig:merit-model}:

\begin{figure}[H]

{\centering \includegraphics[width=0.75\linewidth]{../input/images/generalf} 

}

\caption{Model of perception and preferences for meritocracy and non-meritocracy}\label{fig:merit-model}
\end{figure}

The columns Perceptions and Preferences represent the distinction between this two concepts, usually confused under the label ``beliefs''. Perceptions refers to the extent to which people see that meritocracy functions or applies in their society, which in terms of measurement relates to items such as ``I think hard work is important to get ahead in society'', whereas preferences refer to normative expectations that are usually linked to a ``should'' expression (e.g.~whether hard work should be related to payment). The rows in the table of Figure 1 consider the distinction between meritocratic and non-meritocratic dimensions (Reynolds and Xian \protect\hyperlink{ref-reynolds_perceptions_2014}{2014}). This aspect has been usually treated as different ends of a same continuum in part of the previous research, an assumption that requieres empirical scrutiny. These non-meritocratic elements usually refer to the use of personal contacts or family advantages to get ahead in life.

Regarding the selection of indicators, most of them are taken or adapted from previous studies for the sake of comparability. For meritocratic indicators we use effort and talent as the main components of the traditional concept of merit as defined by Young (\protect\hyperlink{ref-young_rise_1962}{1962}), whereas for non-meritocratic dimensions we use having rich parents and good contacts. The description of the specific items is presented in the methodology section.

The research hypotheses behind this conceptualization are the following:

\begin{itemize}
\item
  H1. The perception of meritocracy is a latent variable based on indicators of the importance attributed to talent and the effort to get ahead in life.
\item
  H2. The non-meritocratic perception is a latent variable that derives from two indicators related to the agreement with the statement that people with contacts and rich parents manage to get ahead.
\item
  H3. Meritocratic preferences are a latent variable based on a normative value of effort and talent.
\item
  H4. Non-meritocratic preferences are a latent variable based on the normative value of the use of personal contacts and having rich parents.
\end{itemize}

\hypertarget{methodology}{%
\section{Methodology}\label{methodology}}

\hypertarget{data}{%
\subsection*{Data collection}\label{data}}
\addcontentsline{toc}{subsection}{Data collection}

The data was obtained through an online questionnaire which was part of a larger study on meritocracy and preferences developed in Chile in 2019 and funded by the national scientific agency FONDECYT. The questionnaire was programmed in Qualtrics and the fieldwork was in charge of an external online survey agency (\href{www.netquest.cl}{netquest.cl}) during December 2019 and January 2020. The sample was selected from a non-probabilistic design in three large cities in Chile. The quota method based on age, sex and educational level was used. Quotas were generated based on the survey of the Public Studies Center (CEP, 2019), which has a high prestige in the country and is also the counterpart agency of ISSP (International Social Survey Programme) in Chile. A total sample of 2,141 people was collected, excluding those who did not answer the questions on the scale and those who did not accept informed consent. As it usually occurs online samples, there were some limitations in achieving the quotas for lower educational levels.

\hypertarget{instrument-design}{%
\subsection{Instrument design}\label{instrument-design}}

The proposed scale of perceptions and preferences about meritocracy consists of 8 indicators that are grouped into the 4 dimensions referred above: Perceptions (meritocratic/non-meritocratic) and preferences (meritocratic/non-meritocratic). In order to achieve at least some comparability with previous studies, the questions were adapted from the items battery ``reasons to get ahead'' (ISSP/GSS), which are mostly used for operationalizing meritocracy in previous empirical studies (Mijs \protect\hyperlink{ref-mijs_paradox_2019}{2019}; Duru-Bellat and Tenret \protect\hyperlink{ref-duru-bellat_whos_2012}{2012}; Reynolds and Xian \protect\hyperlink{ref-reynolds_perceptions_2014}{2014}). The eight items ordered according to dimensions are presented in Table \ref{tab:table-indicadores}. These 8 likert-type items have 5 response alternatives ranging from ``Completely disagree''(1) to ``Completely agree'' (5).

\begin{table}[!h]

\caption{\label{tab:table-indicadores}Items according to dimension.}
\centering
\resizebox{\linewidth}{!}{
\fontsize{10}{12}\selectfont
\begin{tabu} to \linewidth {>{\raggedright\arraybackslash}p{1.5cm}>{\raggedright\arraybackslash}p{2 cm}>{\raggedright}X>{\raggedright}X}
\toprule
Dimension & Factor & Statement (english) & Statement (spanish)\\
\midrule
 &  & Those who try harder get greater rewards than those who work less. & Quienes más se esfuerzan logran obtener mayores recompensas que quienes se esfuerzan menos.\\
\cmidrule{3-4}
 & \multirow{-2}{2 cm}{\raggedright\arraybackslash Meritocratic} & Those who have more talent achieve greater rewards than those who have less talent. & Quienes poseen más talento logran obtener mayores recompensas que quienes poseen menos talento.\\
\cmidrule{2-4}
 &  & Those who have rich parents succeed. & Quienes tienen padres ricos logran salir adelante.\\
\cmidrule{3-4}
\multirow{-4}{1.5cm}{\raggedright\arraybackslash Perception} & \multirow{-2}{2 cm}{\raggedright\arraybackslash Non meritocratic} & Those who have good contacts succeed. & Quienes tienen buenos contactos logran salir adelante.\\
\cmidrule{1-4}
 &  & Those who try harder should get greater rewards than those who work less. & Quienes más se esfuerzan deberían obtener mayores recompensas que quienes se esfuerzan menos.\\
\cmidrule{3-4}
 & \multirow{-2}{2 cm}{\raggedright\arraybackslash Meritocratic} & Those who have more talent should get greater rewards than those who have less talent. & Quienes poseen más talento deberían obtener mayores recompensas que quienes poseen menos talento.\\
\cmidrule{2-4}
 &  & It's fine that those with rich parents get ahead. & Está bien que quienes tienen padres ricos salgan adelante.\\
\cmidrule{3-4}
\multirow{-4}{1.5cm}{\raggedright\arraybackslash Preference} & \multirow{-2}{2 cm}{\raggedright\arraybackslash Non meritocratic} & It's fine that those who have good contacts get ahead. & Está bien que quienes tienen buenos contactos salgan adelante.\\
\bottomrule
\end{tabu}}
\end{table}

\pagebreak

\hypertarget{administration-sets}{%
\subsection{Administration sets}\label{administration-sets}}

With the objective of evaluating the effect of the indicators ordering, the respondents (\emph{n = 2141}) were randomly divided into three different order versions as explained in Figure \ref{fig:appmod} . The scale was presented to the first group (\emph{n = 712}) in the order that appears in Table 2. For the second group (\emph{n = 717}), the order of the items was organized according to the topics of the items, e.g.~for the topic of hard work the item about perception was followed by the item about preference, and te same for the rest of the topics. Finally, for the third group (\emph{n = 712}) the items were completely randomized.

\begin{figure}[H]

{\centering \includegraphics[width=0.75\linewidth]{../input/images/app_mod} 

}

\caption{Survey flow}\label{fig:appmod}
\end{figure}

\hypertarget{methods}{%
\section{Methods}\label{methods}}

For testing the scale's underlying constructs we estimate confirmatory factor analysis models (CFA). The model estimates one factor for each dimension, as represented in the following figure:

\begin{figure}[H]

{\centering \includegraphics[width=0.75\linewidth]{../output/images/meas01} 

}

\caption{Theorical model}\label{fig:meas01}
\end{figure}

CFA was conducted using the \texttt{lavaan} package (version 0.6-3; Rosseel, 2020) with diagonally weighted least squares (DWLS) estimation due to the items' ordinal level of measurement (Kline, 2016; Rosseel, 2020). As recommended by Brown (2008), we assessed model fit by jointly considering the comparative fit index and Tucker-Lewis Index (CFI and TLI; acceptable fit \textgreater{} 0.95), Root of the average squared residual approximation (RMSEA; acceptable fit \textless{} 0.08), Chi-square: (p-value; acceptable fit \textgreater{} 0.05, and Chi-square ratio:\textgreater{} 3).

The study was pre-registered \ldots{} (link) ***

\hypertarget{results}{%
\section{Results}\label{results}}

\hypertarget{descriptive-analysis}{%
\subsection{Descriptive analysis}\label{descriptive-analysis}}

Como puede observarse en la tabla \ref{tab:desc01}, los indicadores poseen valores que van desde el 1 (totalmente en desacuerdo) al 5 (totalmente de acuerdo). Se observan promedios desde 2.41 correspondiente a preferencia-contactos, hasta 3.89 correspondiente a la preferencia-esfuerzo. Ambos indicadores son coherentes con la adhesión general a la meritocracia, privilegiando aspectos individuales como el esfuerzo.

\begin{table}[!h]

\caption{\label{tab:desc01}Descriptive statistics of the scale.}
\centering
\fontsize{10}{12}\selectfont
\begin{tabular}[t]{lllll}
\toprule
  & Mean & SD & Min & Max\\
\midrule
A.Perception Effort & 3.20 & 1.38 & 1 & 5\\
B.Perception Talent & 3.02 & 1.16 & 1 & 5\\
C.Perception rich parents & 3.66 & 1.36 & 1 & 5\\
D.Perception contacts & 3.79 & 1.24 & 1 & 5\\
E.Preferences Effort & 3.89 & 1.25 & 1 & 5\\
F.Preferences Talent & 3.24 & 1.19 & 1 & 5\\
G.Preferences rich parents & 2.69 & 1.18 & 1 & 5\\
H.Preferences contacts & 2.41 & 1.11 & 1 & 5\\
\bottomrule
\end{tabular}
\end{table}

El gráfico XX muestra \ldots{}
A partir del la Figura \ref{fig:plotlikert} se observa que en general hay una mayor percepción de elementos no meritocráticos que meritocráticos, mientras que en el caso de preferencias ocurre lo opuesto. En cuanto a las preferencias, llama la atención el rol preponderante del esfuerzo por sobre el talento.

\begin{figure}[H]

{\centering \includegraphics[width=0.75\linewidth]{../output/images/plotlikert} 

}

\caption{Descriptive plot}\label{fig:plotlikert}
\end{figure}

El Grafico \ref{fig:corpoly} muestra \ldots.
Se observan relaciones de moderada a alta intensidad entre los indicadores que corresponden al mismo factor (por ejemplo, percepción de meritocracia esfuerzo con talento r=0.56), mientras que se observan correlaciones de baja intensidad entre el resto de los cruces. Destacan además las relaciones entre las percepciones y preferencias de tipo meritocráticas, lo que no ocurre con los indicadores no meritocraticos.

\begin{figure}[H]

{\centering \includegraphics[width=0.75\linewidth]{../output/images/corpoly} 

}

\caption{Polychoric correlation plot}\label{fig:corpoly}
\end{figure}

En suma, los análisis descriptivos señalan una relativa adhesión a la moral meritocratica, las cuales, a modo general, se expresan en una mayor preferencia por criterios meritocraticos y una menor por no meritocráticos. Igualmente, es observable una relativamente baja percepción de meritocracia. Se observa además una relación coherente entre los indicadores según lo propuesto en el modelo teórico en el preregistro del estudio, es decir, los pares de items asociados a un factor específico muestran correlaciones con un tamaño de efecto grande (por ejemplo, preferencias meritocrácticas por los items asociados a esfuerzo y a talento). En particular las asociaciones entre esfuerzo y talento son relevantes, ya que desestiman previos supuestos sobre que el talento no sería un criterio meritocrático (Mijs \protect\hyperlink{ref-mijs_paradox_2019}{2019}), de otra manera la correlación sería cero o negativa. Junto a ello, vemos que no existe una correlacion negativa entre aspectos meritocráticos y no meritocráticos, desestimando los supuestos de estudios previos que señalaban que estas dimensiones serían los polos opuestos de un mismo continuo (Reynolds and Xian \protect\hyperlink{ref-reynolds_perceptions_2014}{2014}).

\hypertarget{confirmatory-factor-analysis}{%
\subsection{Confirmatory Factor Analysis}\label{confirmatory-factor-analysis}}

This section tests the item's battery and the corresponding conceptual framework under test. For this, we first estimate a confirmatory factor analysis model for the whole sample, and secondly we test the order effects applying the same model to each of the three order versions.

\hypertarget{full-sample-cfa}{%
\subsubsection{Full sample CFA}\label{full-sample-cfa}}

Figure X shows the results of the estimation for the four-factor model:

\begin{quote}
Julio: Esta sección debe ser revisada:
\end{quote}

\begin{itemize}
\tightlist
\item
  CFA general
\item
  Fit x modelo
\end{itemize}

{[}Aquí SEM plot con cargas y ajuste global del modelo{]}

Los modelos confirmatorios fueron estimados en primer lugar para cada uno de los tres administration-sets. Los resultados se presentan en la Tabla \ref{tab:tab-loads}

\begin{table}[!h]

\caption{\label{tab:tab-loads}Factor loads and model fit.}
\centering
\resizebox{\linewidth}{!}{
\fontsize{9}{11}\selectfont
\begin{threeparttable}
\begin{tabular}[t]{lcccccccccccc}
\toprule
\multicolumn{1}{c}{ } & \multicolumn{12}{c}{Factor loadings} \\
\cmidrule(l{3pt}r{3pt}){2-13}
\multicolumn{1}{c}{ } & \multicolumn{4}{c}{Version 1} & \multicolumn{4}{c}{Version 2} & \multicolumn{4}{c}{Version 3} \\
\cmidrule(l{3pt}r{3pt}){2-5} \cmidrule(l{3pt}r{3pt}){6-9} \cmidrule(l{3pt}r{3pt}){10-13}
Variables & 1 & 2 & 3 & 4 & 1 & 2 & 3 & 4 & 1 & 2 & 3 & 4\\
\midrule
Perception Effort & 0.69 &  &  &  & 0.76 &  &  &  & 0.70 &  &  & \\
Perception Talent & 0.81 &  &  &  & 0.72 &  &  &  & 0.65 &  &  & \\
Perception rich parents &  & 0.85 &  &  &  & 0.84 &  &  &  & 0.81 &  & \\
Perception contacts &  & 0.94 &  &  &  & 0.81 &  &  &  & 0.89 &  & \\
Preferences Effort &  &  & 0.85 &  &  &  & 0.82 &  &  &  & 0.66 & \\
Preferences Talent &  &  & 0.64 &  &  &  & 0.65 &  &  &  & 0.59 & \\
Preferences rich parents &  &  &  & 0.55 &  &  &  & 1.04 &  &  &  & 0.78\\
Preferences contacts &  &  &  & 1.26 &  &  &  & 0.52 &  &  &  & 0.77\\
\hline
\hspace{1em}$\chi^2\text{(df)}$ &  & 42.3(14) &  &  &  & 107.6(14) &  &  &  & 63.3(14) &  & \\
\hspace{1em}$\text{CFI}$ &  & 0.993 &  &  &  & 0.961 &  &  &  & 0.979 &  & \\
\hspace{1em}$\text{RMSEA}$ &  & 0.053 &  &  &  & 0.097 &  &  &  & 0.070 &  & \\
\hspace{1em}$N$ &  & 712 &  &  &  & 717 &  &  &  & 712 &  & \\
\bottomrule
\end{tabular}
\begin{tablenotes}
\item \textit{Note: } 
\item Standardized factor loadings using DWLS estimator ; CFI = Comparative fit index (scaled); RMSEA = Root mean square error of approximation (scaled)
\end{tablenotes}
\end{threeparttable}}
\end{table}

Como se puede observar en la Tabla XX, todos los modelos independiente de los órdenes obtuvieron un ajuste adecuado. con CFI superiores a .95 y RMSEA inferiores a .8, por contraparte ni un modelo logró un chi-square no significativo, aunque tanto el modelo aleatorio como el primero obtuvieron un adecuado chi-square ratio menor a 3. El primero de los órdenes fue el que obtuvo mejor ajuste (CFI=0.998, TLI= 0.995,RMSEA=0.037, \(\chi2\)(df=14)=28,03, p = 0.014), seguido por el orden aleatorio de los ítems (CFI=0.992, TLI=0.984,RMSEA=0.051, \(\chi2\)(df=14)=39.09, p \textless{} 0.001). Por su parte, la escala ordenada por temáticas parece generar un efecto framing en el cual la relación entre las percepciones y las preferencias parecen sobreestimadas, afectando de esta manera el ajuste (CFI=0.984, TLI=0.968,RMSEA=.071, \(\chi2\)(df=14)=64.156, p \textless{} 0.001).

Si bien todas las pruebas obtuvieron indicadores relativamente adecuados, las diferencias mencionadas entre los ajustes de los modelos según orden son estadísticamente significativas, lo cual fue evaluado a partir de un análisis de invarianza. Se concluye que entre los tres órdenes no existe invarianza configuracional, es decir, no poseen la misma dimensionalidad y por ello no ajustan igualmente al modelo teórico ({\textbf{???}}). Esto se debe al efecto producido por la aparición conjunta de los ítems de un mismo factor en el orden 1, lo cual aumenta el ajuste del modelo. Además, en el orden 2, al preguntarse seguidamente por la percepción y la preferencia en torno al mismo indicador, aumentan las relaciones cruzadas disminuyendo el ajuste del modelo. De modo coherente, el modelo aleatorio presentó un ajuste intermedio entre el orden 1 y el orden 2.

\pagebreak

\begin{quote}
Julio: ¿Esto se debe mantener?
\end{quote}

\begin{longtable}[]{@{}cccc@{}}
\caption{Items according to dimension.}\tabularnewline
\toprule
& Contrast Model 1 Factor & theoretical model 4 Factors & Model with M.I.\tabularnewline
\midrule
\endfirsthead
\toprule
& Contrast Model 1 Factor & theoretical model 4 Factors & Model with M.I.\tabularnewline
\midrule
\endhead
\textbf{n} & 1769 & 1769 & 1769\tabularnewline
\textbf{CFI} & 0.595 & 0.988 & 0.994\tabularnewline
\textbf{TLI} & 0.433 & 0.976 & 0.985\tabularnewline
\textbf{RMSEA} & 0.226 & 0.047 & 0.036\tabularnewline
\textbf{\(\chi2\)} & 1830.839 & 68.661 & 40.250\tabularnewline
\textbf{p} & .000 & .000 & .000\tabularnewline
\textbf{\(\chi2\)/df} & 65.38 & 4.90 & 3.32\tabularnewline
\bottomrule
\end{longtable}

EL modelo teórico propuesto de cuatro factores ajustó, como se observa en la tabla 4, mejor que el modelo de contraste de 1 factor. El modelo teórico ajustó de manera relativamente adecuada, pues muestra indicadores óptimos para CFI= 0.987, TLI = 0.975 y RMSEA=.041, aunque posee indicadores deficientes para la prueba \(\chi2\)(df=14)=67.6, p-value=.000. Para evaluar posibles mejoras de la escala, se analizaron las relaciones propuestas por los índices de modificación. Estos indican la existencia de dos cargas cruzadas no especificadas. Cuando se generó un modelo siguiendo estas recomendaciones, hubo una mejora considerable del modelo, aunque el nuevo modelo tampoco obtuvo un \(\chi2\) ratio menor a 3 y obtuvo cargas factoriales muy bajas (\(\tau\))\textless{} 0.15, por lo tanto, siguiendo las recomendación de Brown (2006) de solo aceptar las propuesta de los índices de modificación cuando se posee teoria y evidencia sólida, se ha decidido no incorporar estos parámetros al modelo.

\hypertarget{conclusiuxf3n}{%
\section{Conclusión}\label{conclusiuxf3n}}

Considerando la ventaja del orden, respecto aleatorio despeja la posibilidad del efecto framing, es decir, de que el resultado del modelo se deba al orden de las preguntas, se ha decidido seguir utilizando el orden 3.

uso conjunto percepciones-preferencias

\pagebreak

\hypertarget{references}{%
\section{References}\label{references}}

\hypertarget{refs}{}
\leavevmode\hypertarget{ref-alesina_Fairness_2005}{}%
Alesina, A., and G. Angeletos. 2005. ``Fairness and Redistribution.'' \emph{American Economic Review}, 960--80.

\leavevmode\hypertarget{ref-arrow_meritocracy_2000}{}%
Arrow, Kenneth J., Samuel Bowles, and Steven N. Durlauf, eds. 2000. \emph{Meritocracy and Economic Inequality}. Princeton, N.J: Princeton University Press.

\leavevmode\hypertarget{ref-breenClassInequalityMeritocracy1999}{}%
Breen, Richard, and John H. Goldthorpe. 1999. ``Class Inequality and Meritocracy: A Critique of Saunders and an Alternative Analysis1.'' \emph{The British Journal of Sociology} 50 (1): 1--27. \url{https://doi.org/10.1111/j.1468-4446.1999.00001.x}.

\leavevmode\hypertarget{ref-davey_preference_1999}{}%
Davey, L. M., D. R. Bobocel, L. S. Son Hing, and M. P. Zanna. 1999. ``Preference for the Merit Principle Scale: An Individual Difference Measure of Distributive Justice Preferences.'' \emph{Social Justice Research} 12 (3): 223--40.

\leavevmode\hypertarget{ref-dimick_Models_2018}{}%
Dimick, Matthew, David Rueda, and Daniel Stegmueller. 2018. ``Models of Other-Regarding Preferences, Inequality, and Redistribution.'' \emph{Annual Review of Political Science} 21 (1): 441--60. \url{https://doi.org/10.1146/annurev-polisci-091515-030034}.

\leavevmode\hypertarget{ref-duru-bellat_whos_2012}{}%
Duru-Bellat, Marie, and Elise Tenret. 2012. ``Who's for Meritocracy? Individual and Contextual Variations in the Faith.'' \emph{Comparative Education Review} 56 (2): 223--47. \url{https://doi.org/10.1086/661290}.

\leavevmode\hypertarget{ref-goldthorpe_myth_2003}{}%
Goldthorpe, John. 2003. ``The Myth of Education-Based Meritocracy.'' \emph{New Economy} 10 (4): 234--39. \url{https://doi.org/10.1046/j.1468-0041.2003.00324.x}.

\leavevmode\hypertarget{ref-hadjar_meritokratie_2008}{}%
Hadjar, Andreas. 2008. \emph{Meritokratie Als Legitimationsprinzip}. Wiesbaden: VS Verlag.

\leavevmode\hypertarget{ref-kunovich_systems_2007}{}%
Kunovich, Sheri, and Kazimierz M. Slomczynski. 2007. ``Systems of Distribution and a Sense of Equity: A Multilevel Analysis of Meritocratic Attitudes in Post-Industrial Societies.'' \emph{European Sociological Review} 23 (5): 649--63. \url{https://doi.org/10.1093/esr/jcm026}.

\leavevmode\hypertarget{ref-landWeSatTable2006}{}%
Land, Hilary. 2006. ``We Sat down at the Table of Privilege and Complained About the Food \textsuperscript{1}.'' \emph{The Political Quarterly} 77 (s1): 45--60. \url{https://doi.org/10.1111/j.1467-923X.2006.00780.x}.

\leavevmode\hypertarget{ref-MadeiraPrimesConsequencesSystematic2019}{}%
Madeira, Ana Filipa, Rui Costa-Lopes, John F. Dovidio, Gonçalo Freitas, and Mafalda F. Mascarenhas. 2019. ``Primes and Consequences: A Systematic Review of Meritocracy in Intergroup Relations.'' \emph{Frontiers in Psychology} 10 (September). \url{https://doi.org/10.3389/fpsyg.2019.02007}.

\leavevmode\hypertarget{ref-markovits_Meritocracy_2019}{}%
Markovits, Daniel. 2019. \emph{The Meritocracy Trap: How America's Foundational Myth Feeds Inequality, Dismantles the Middle Class, and Devours the Elite}. New York: Penguin Press.

\leavevmode\hypertarget{ref-mijs_paradox_2019}{}%
Mijs, Jonathan J B. 2019. ``The Paradox of Inequality: Income Inequality and Belief in Meritocracy Go Hand in Hand.'' \emph{Socio-Economic Review}, January. \url{https://doi.org/10.1093/ser/mwy051}.

\leavevmode\hypertarget{ref-newman_false_2015}{}%
Newman, Benjamin J., Christopher D. Johnston, and Patrick L. Lown. 2015. ``False Consciousness or Class Awareness? Local Income Inequality, Personal Economic Position, and Belief in American Meritocracy.'' \emph{American Journal of Political Science} 59 (2): 326--40. \url{https://doi.org/10.1111/ajps.12153}.

\leavevmode\hypertarget{ref-reynolds_perceptions_2014}{}%
Reynolds, Jeremy, and He Xian. 2014. ``Perceptions of Meritocracy in the Land of Opportunity.'' \emph{Research in Social Stratification and Mobility} 36 (June): 121--37. \url{https://doi.org/10.1016/j.rssm.2014.03.001}.

\leavevmode\hypertarget{ref-saundersMightBritainBe1995}{}%
Saunders, Peter. 1995. ``Might Britain Be a Meritocracy?'' \emph{Sociology} 29 (1): 23--41. \url{https://doi.org/10.1177/0038038595029001003}.

\leavevmode\hypertarget{ref-schroder_Income_2017}{}%
Schröder, Martin. 2017. ``Is Income Inequality Related to Tolerance for Inequality?'' \emph{Social Justice Research} 30 (1): 23--47. \url{https://doi.org/10.1007/s11211-016-0276-8}.

\leavevmode\hypertarget{ref-son_hing_merit_2011-1}{}%
Son Hing, Leanne S., D. Ramona, Mark P. Zanna, Donna M. Garcia, Stephanie S. Gee, and Katie Orazietti. 2011. ``The Merit of Meritocracy.'' \emph{Journal of Personality and Social Psychology} 101 (3): 433--50. \url{https://doi.org/10.1037/a0024618}.

\leavevmode\hypertarget{ref-yairMeritocracy2007}{}%
Yair, Gad. 2007. ``Meritocracy.'' In \emph{The Blackwell Encyclopedia of Sociology}, edited by George Ritzer. Oxford, UK: John Wiley \& Sons, Ltd. \url{https://doi.org/10.1002/9781405165518.wbeosm082}.

\leavevmode\hypertarget{ref-young_rise_1962}{}%
Young, M. 1962. \emph{The Rise of the Meritocracy}. Baltimore: Penguin Books.

\leavevmode\hypertarget{ref-youngRiseMeritocracy1994}{}%
Young, Michael Dunlop. 1994. \emph{The Rise of the Meritocracy}. New Brunswick, N.J., U.S.A: Transaction Publishers.

\hypertarget{appendix}{%
\section*{Appendix}\label{appendix}}
\addcontentsline{toc}{section}{Appendix}

\begin{longtable}[]{@{}lcc@{}}
\caption{Representativeness of the sample.}\tabularnewline
\toprule
& Sample & CEP\tabularnewline
\midrule
\endfirsthead
\toprule
& Sample & CEP\tabularnewline
\midrule
\endhead
\textbf{Gender} & &\tabularnewline
Men & 49,82\% & 50,52\%\tabularnewline
Women & 50.18\% & 49,47\%\tabularnewline
\textbf{Age} & &\tabularnewline
18 - 24 & 18,55\% & 18,17\%\tabularnewline
25 - 34 & 18,86\% & 17,48\%\tabularnewline
35 - 44 & 19.09\% & 19,98\%\tabularnewline
45 - 54 & 17,96\% & 19,23\%\tabularnewline
55 - or more & 25,54\% & 25.11\%\tabularnewline
\textbf{Education} & &\tabularnewline
Primary or less & 2,93\% & 15,88\%\tabularnewline
Hig school & 43,23\% & 37,04\%\tabularnewline
Non university & 32,63\% & 28,93\%\tabularnewline
university or more & 21,21\% & 18,13\%\tabularnewline
\bottomrule
\end{longtable}

\end{document}
